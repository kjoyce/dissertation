%   Your header stuff goes here
\usepackage{natbib}
\usepackage{amsmath}
\usepackage{amsthm}
\usepackage{amssymb}
\usepackage{hyperref}
\usepackage{cleveref}
\usepackage[framed]{mcode}
\usepackage{tikz}
\usepackage{units}

\usetikzlibrary{matrix}
\usetikzlibrary{shapes.geometric}
\usetikzlibrary{shapes.misc}
\usetikzlibrary{decorations.pathmorphing}
\usetikzlibrary{arrows}
\usetikzlibrary{backgrounds}
\usetikzlibrary{positioning}
\usetikzlibrary{fadings}
\DeclareMathOperator*{\argmin}{arg\,min}
\DeclareMathOperator*{\supp}{supp}
\DeclareMathOperator*{\acos}{acos}

\theoremstyle{plain}
\newtheorem{thm}{Theorem}[section]
\newtheorem{lem}[thm]{Lemma}
\newtheorem{prop}[thm]{Proposition}
\newtheorem{cor}[thm]{Corollary}
\newtheorem{defn}[thm]{Definition}

\renewcommand{\bar}{\overline}
\renewcommand{\hat}{\widehat}
\renewcommand{\tilde}{\widetilde}
\newcommand{\eps}{\varepsilon}
\newcommand{\del}{\partial}
\newcommand{\RR}{\ensuremath{\mathbb R}}
\newcommand{\vect}[1]{\boldsymbol{#1}}
\newcommand{\DD}{\ensuremath{\mathfrak D}}
\newcommand{\K}{\ensuremath{\mathscr K}}
\newcommand{\R}{\ensuremath{\mathscr R}}
\newcommand{\HH}{\ensuremath{\mathscr H}}

