\documentclass[12pt]{article}
\usepackage{amsmath, amssymb, amsfonts}
\renewcommand{\abstract}[4]{{\noindent\it #3}\\{\bf #2}\\[6pt] {\upshape #4}}
\newcommand{\presenter}{$^{\dagger}$}
\newcommand{\gradstudent}{$^{**}$}
\newcommand{\undergrad}{$^{*}$}

\begin{document}
%
% This is an abstract template for inclusion in the PNW MAA meeting program.
%
% Standard LaTeX (including AMS-LaTeX commands) are allowed.
%
% Please submit the .tex file (not a dvi or pdf file)
%
% Abstract format is as follows:
%
% \abstract{presenternickname} %
% {First Author Name, FA Institution\presenter\\Second Author Name, SA Institution\\etc.}
% {Paper Title}
% {Text of the paper abstract}
%
% Remark: in the author line \presenter, \gradstudent, and \undergrad
% denote the appropriate role. Multiple roles are allowed.
%
% Two examples follow:
%
%\abstract{professionals}
%{Bill Pompous, Big U\\Sherry Professional, Private College\presenter}
%{Partial Homomorphisms of Deflators}
%{A very important talk with lots of deep results.}

\abstract{pompousb} %nicknames are used by the Program Chair; it's helpful if you use your last name+initial, e.g. pompousb
{% AUTHORS
Kevin Joyce, University of Montana\\ 
John Bardsley, University of Montana\\
Aaron Luttman, National Securities Technologies
}
{% TITLE
Enhanced Gibbs sampling for an application to X-ray imaging
}
{% ABSTRACT
Image deblurring techniques derived from convolution require, a priori, an estimate for the convolution kernel or point spread function (PSF).
Standard techniques for estimating the PSF involve imaging a bright point source, but this is not always feasible (e.g.~high energy radiography).  
This work takes a novel non-parametric approach to modeling a radially symmetric PSF, in which an estimate can be obtained from the calibration image of a vertical edge.
Moreover, we take a hierarchical Bayesian approach that in addition to providing a method for estimation, also gives a quantification of uncertainty in the estimate.

We will present a recently developed improvement to the Gibbs algorithm for simulating samples of the Bayesian posterior of the hierarchical model, referred to as partial collapse.
The improved algorithm has been independently derived in several other works, however, it has been shown that partial collapse may be improperly implemented resulting in a sampling algorithm that that no longer converges to the desired posterior.
The algorithm we present is proven to satisfy invariance with respect to the target density.

}

% Please name your file using the presenter's last name and first
% initial; if the presenter is submitting more than one abstract,
% enumerate, as in:
% pompousb.tex and pompousb2.tex
%
\end{document}
