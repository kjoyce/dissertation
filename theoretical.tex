\setlength{\parindent}{2ex}
%\newcommand{\mathscr}{\mathcal}
\begin{chapter}{Reconstruction on the Continuum}\label{chapter:theoretical}

The goal of this section is to develop the necessary mathematical tools to encapsulate prior notions of smoothness and radial symmetry within the structure of appropriately developed Hilbert spaces.
This is done within the framework of distributions from linear PDE theory \cite{hormander1983,rudin1991,griffel2002}. %and we establish a one-to-one correspondence between the Hilbert space of radially symmetric PSFs and the Hilbert space of their radial representations.  
%This correspondence leads to an alternate formulation of the inverse problem on the radial representations.
We will first establish the relevant background from that theory.
Then, we define symmetry in terms of pre-composition with a smooth map that is many-to-one on circles.  
This defines a pullback mapping that will ultimately establish the space of radially symmetric functions, the space of their radial representations, and an isometry between them.  

\subsection{Distributional Hilbert Spaces}
We first establish notation and preliminaries from the theory of distributions.
For a complete development, see one of various classical texts on the subject, such as \cite{hormander1983,rudin1991}.

%For $\Omega$ an open subset of $\R^n$, the space of compactly supported infinitely differentiable functions defined on $\Omega$ is the space of test functions on $\Omega$ and we denote them as $\DD(\Omega)$.
Let $\DD(\Omega)$ denote compactly supported smooth test functions defined on an open set $\Omega$.
We adopt the notation $\langle f, \phi\rangle$ for the action of a linear functional $f$ on $\phi \in \DD(\Omega)$. 
The space of continuous linear functionals, or distributions, on $\DD(\Omega)$ are denoted with $\DD^*(\Omega)$. 
Continuity is in the sense that for every compact set $K \subset \Omega$, there exists $C$ and $k$ so that
\begin{equation}
  |\langle u, \phi \rangle | \le C \sum_{|\alpha| \le k} \sup |\del^\alpha \phi|,
\end{equation} 
for any multi-index $\alpha$.
A functional $f$ is continuous if and only if for every sequence $\{\phi_j\}$ in $\DD(\Omega)$ converging to zero in the sense that for every fixed $\alpha$, $\sup |\del^\alpha \phi_j| \to 0$ with some fixed compact $K \subset \Omega$, such that $\supp \phi_j \subseteq K$ for all $j$, then $\langle f, \phi_j\rangle \to 0$ \cite{hormander1983}.
We freely use the natural inclusion of functions $g \to \tilde g \in \DD^*(\Omega)$ by $\langle \tilde g, \phi \rangle = \int g\, \phi\,dx$ when the integration exists and omit the tilde notation distinguishing $g$ and $\tilde g$, as the representation should be clear from context. 

The space of square Lebesgue integrable functions on $\Omega$, $L^2(\Omega)$, embeds naturally into $\DD^*(\Omega)$.
This is done by taking sequences of test functions $\{\phi_n\}$ such that $\int \phi_n \bar{\phi_n}\,dx$ defines a Cauchy sequence in $\R^+$, and defining $\langle \phi, \psi\rangle := \lim_{n\to\infty} \langle \phi_n,\psi\rangle$. It can be shown that $\phi \in D^*(\Omega)$ and that sequences of test functions are sufficient for representing all elements of $L^2(\Omega)$. Moreover, it can be shown that
\begin{equation} 
  (\phi,\psi)_{L_2(\Omega)} := \lim_{n\to\infty} \int \phi_n,\psi_n\,dx
\end{equation}
defines an inner product agrees with the usual $L^2$ inner product.

In a similar way, the Sobolev space $\mathscr H^m$ embeds in $\DD^*(\Omega)$ by considering Cauchy functions with respect to inner product $\sum_{|\alpha|\le m}(\del^\alpha \cdot\,, \del^\alpha \cdot \,)_{L^2}$ where the sum is over all multi-indices $\alpha$.
These spaces form a sequence of closed subspaces $L^2(\Omega) = \mathscr H^0(\Omega) \supset \mathscr H^1(\Omega) \supset \dots \supset \mathscr H^m(\Omega) \supset\dots \supset \DD(\Omega)$ \cite{hutson1980}.
In our development, we will be primarily concerned with open subsets of $\R^2$ and $\R$, and we denote such subsets as $\Omega_2$ and $\Omega_1$ respectively.

A standard result from distribution theory \cite[Theorem 4.1.5]{hormander1983} states 
\begin{thm} \label{thm:testFunDensity}
  For $\Omega$ open in $\R^n$, if $f \in \DD^*(\Omega)$, then there is a sequence of test functions $\{f_n\}\subset \DD(\Omega)$ such that $\langle f_n,\phi\rangle \to \langle f,\phi\rangle$ for all $\phi \in \DD(\Omega)$.
\end{thm}

%\Cref{thm:testFunDensity} will allow us to define a property of $\DD^*(\Omega)$ by establishing it on $\DD(\Omega)$.
\Cref{thm:testFunDensity} is used in \cite{hormander1983} to establish many of the computation rules involving derivatives and, in particular, the various chain rules associated with differential forms are established on $\DD(\Omega)$ and then extended continuously to $\DD^*(\Omega)$ by a computation that expresses the action on test functions, such as integration by parts. 
This is the general strategy we will use to establish many of the results in this section.

%%%%%%%%%%%%%%%%%%%%%%%%%%%%%%%%%%%%%%%%%%%%%%%%%%%%%%%%%%%%%%%%%%%%%%%%%%%%%%%%
%                                                                              % 
%                             pullback operator                                % 
%                                                                              % 
%%%%%%%%%%%%%%%%%%%%%%%%%%%%%%%%%%%%%%%%%%%%%%%%%%%%%%%%%%%%%%%%%%%%%%%%%%%%%%%% 
\subsection{Radial Symmetry via \textcolor{red}{The} Pullback Operator} \label{sec:pullback}
Symmetry is established by casting it in terms of a many-to-one smooth map, $T$, that is constant on circles of a fixed radius.  
%$T(x,y) = h(x^2 + y^2)$, where $h$ is an increasing diffeomorphism of $(0,\infty)$ to itself. 
%Typically $h$ is the square-root function.
%Different choices of $h$ lead to non-uniform grid spacings for the reconstruction of $\rho$ and will be explored in more detail in section \ref{psf_reconstruction}.
When $f$ is a function, it has radial symmetry if there exists a function $\rho$, so that $f = \rho \circ T$.
This notion is easily adapted to distributions by developing the corresponding linear pullback operator to $T$ that maps $\rho$ to $f$.
%With this viewpoint, the relationship between the inverse problem in terms of the two dimensional representation and the radial representation is given explicitly through an isometry.
In this subsection, we will explicitly construct $T$ and show that it establishes a one-to-one correspondence. 
Using this correspondence, we will define the space of radially symmetric distributions and the space of their radial representations.

\begin{thm} \label{thm:pullback}
  Let $\Omega_1 := (0,\infty) \subset \R$ and $\Omega_2 := \R^2 \setminus \{x = 0\text{ or }y=0\}$.
  For $h:\Omega_1\to\Omega_1$ an increasing diffeomorphism, define $T:\Omega_2 \to \Omega_1$ by $T(x,y) = h(x^2 + y^2)$, then there exists a unique, injective, continuous, linear operator $T^\sharp:\DD^*(\Omega_1) \to \DD^*(\Omega_2)$ so that $\langle T^\sharp \rho ,\phi\rangle = \langle \rho \circ T,\phi\rangle$ for all $\phi \in \DD(\Omega_2)$ and $\rho \in \DD(\Omega_1)$.
\end{thm}

To prove this result, we first establish a straight-forward, yet technical, computation in the following lemma.

\begin{lem} \label{lem:existence}
  For each $\phi\in \DD(\Omega_2)$ there exists $\psi_{\phi} \in \DD(\Omega_1)$ so that for any $\rho \in \DD(\Omega_1)$
  \begin{equation}
    \langle \rho \circ T, \phi \rangle_{\Omega_2} = \langle \rho, \psi_\phi\rangle_{\Omega_1}.
  \end{equation}
\end{lem}
\begin{proof}
  Let $Q_{ij} = \{ (x,y): (-1)^i x>0, (-1)^jy>0\}$ for $i,j \in \{0,1\}$ so that $\bigcup Q_{ij} = \Omega_2$.
  Define $T_{ij}:Q_{ij} \to R\subset \R^2$ by 
  \begin{equation}
    T_{ij}(x,y) = \Big(T(x,y), (-1)^jy\Big). 
  \end{equation}
  Observe that each $T_{ij}$ is a diffeomorphism onto $R = \left\{(r,t): 0 < t < \sqrt{h^{-1}(r)}\right\}$ with inverse 
  \begin{align}
    &T_{ij}^{-1}(r,t) = \Big((-1)^i\sqrt{h^{-1}(r) - t^2}, (-1)^jt\Big),\\
    \intertext{ and }
    &\left|dT_{ij}^{-1}(r,t)\right| = \frac12 {h^{-1}}'(r)\left(h^{-1}(r) - t^2\right)^{-1/2}. \label{eq:determinant}
  \end{align}
  Furthermore, note that $T \circ T_{ij}^{-1}(r,t) = r$ by the definition of $T_{ij}$.
  Now, given $\rho \in \DD(\Omega_1)$, a change of variables results in
  \begin{align}
    \langle \rho \circ T, \phi\rangle_{\Omega_2} 
%    &= \iint_{\Omega_2} \rho\circ T(x,y) \phi(x,y)dxdy \\
    &= \sum_{ij}\iint_{Q_{ij}} \rho\circ T(x,y)\cdot \phi(x,y)dxdy \nonumber \\
    &= \sum_{ij}\iint_{R} \rho(r)\cdot  \phi \circ T_{ij}^{-1}(r,t)\left|dT_{ij}\right|drdt \nonumber \\
    &= \int_0^\infty \rho(r) \left(\int_0^{\sqrt{h^{-1}(r)}} \sum_{ij}\phi \circ T_{ij}^{-1}(r,t)\left|dT_{ij}\right|dt \right)dr \label{eq:phiEquation}.
%    &=: \int_{\Omega_1} \rho(r) \Phi_{\phi,T_{ij}}(r) dr. 
%    &= \langle \rho, \Phi_{\phi,T_{ij}} \rangle_{\Omega_1},
  \end{align}
  Let 
  \begin{equation}
    \psi_\phi(r) = \int_0^{\sqrt{h^{-1}(r)}} \sum_{ij}\phi \circ T_{ij}^{-1}(r,t)\left|dT_{ij}\right|dt, \label{eq:psiPhiDef}
  \end{equation}
  and we must show that $\psi_\phi \in \DD(\Omega_1)$.
  Note that $\supp \left(\phi \circ T_{ij}^{-1} \right) = T_{ij}( \supp \phi )\subseteq R$, and since $T_{ij}$ is a diffeomorphism, $\phi \circ T_{ij}^{-1} \in \DD(\Omega_2)$. 
%  Moreover, integrating marginally over $t$ and summing over ${i,j}$ are smooth operations, so $\Phi_{\phi,T_{ij}}$ is smooth. 
%  The support of $\Phi_{\phi,T_{ij}}$ is the projection of the support of $\phi\circ T_{ij}^{-1}$, hence is compact.
%  Marginially integrating $\phi\circ T_{ij}^{-1}$ results in a test function, thus $\Phi_{\phi,T_{ij}} \in \DD(\Omega_1)$. 
  Since $\sum \phi\circ T_{ij}^{-1}(r,t)$ is smooth and compactly supported, a standard result from analysis \cite[pg. 433]{strichartz2000} implies that $\psi_\phi(r)$ resulting from integrating in $t$ is a smooth function whose support is the projection of the support of the integrand.
\end{proof}

%As a corollary to this result, we can establish the existence in the statement of \Cref{thm:pullback}. That is, define $T^\sharp:\DD^*(\Omega_1) \to \DD^*(\Omega_2)$ by 
We can now proceed to prove \Cref{thm:pullback}.
\begin{proof}
  Using \Cref{lem:existence}, define
  \begin{equation}
    \langle T^\sharp \rho, \phi \rangle_{\Omega_2} := \langle \rho, \psi_\phi\rangle_{\Omega_1}.
  \end{equation}
  To see that $T^\sharp \rho \in \DD^*(\Omega_2)$, let $\{\phi_n\} \to 0$ in $\DD(\Omega_2)$, so in particular (for $\alpha = 0$), $\sup|\psi_n| \to 0$. 
  Then, by \eqref{eq:psiPhiDef}, $\sup |\phi_{\psi_n}| \to 0$, and thus $\langle \rho, \psi_{\phi_n}\rangle \to 0$. 
  The linearity and continuity of $T^\sharp$ follow directly from the definition.
  That is 
  \begin{equation}
    \langle T^\sharp \rho_1 + T^\sharp \rho_2,\phi\rangle_{\Omega_2} = \langle \rho_1,\psi_\phi\rangle_{\Omega_1} + \langle \rho_2 ,\psi_\phi \rangle_{\Omega_1} = \langle T^\sharp(\rho_1+\rho_2),\psi_\phi\rangle_{\Omega_2}, 
  \end{equation} 
  and if $\langle \rho_n, \psi\rangle \to 0$ for all $\psi \in \DD(\Omega_1)$, then 
  \begin{equation}
    \langle T^\sharp \rho_n,\phi\rangle_{\Omega_2} = \langle \rho_n,\psi_\phi\rangle_{\Omega_1} \to 0.
  \end{equation}

  Uniqueness is a consequence of \Cref{thm:testFunDensity}. That is, suppose $T^\dagger:\DD^*(\Omega_1) \to \DD^*(\Omega_2)$ is a continuous linear functional such that $\langle T^\dagger \rho,\phi \rangle = \langle \rho \circ T, \phi\rangle$ for all $\phi \in \DD(\Omega_2)$ whenever $\rho \in \DD(\Omega_1)$. Then, for any $\rho \in \DD^*(\Omega_1)$, let $\{\rho_n\}\subset \DD(\Omega_1)$ converge to $\rho$ (in the $\DD^*(\Omega_1)$ sense), so 
  \begin{equation}
    \left\langle (T^\sharp - T^\dagger)\rho,\phi\right\rangle_{\Omega_2} = \lim \langle T^\sharp\rho_n,\phi\rangle_{\Omega_2} - \lim \langle T^\dagger\rho_n,\phi\rangle_{\Omega_2} = 0.
  \end{equation}
  Hence $T^\sharp = T^\dagger$.
  We have established all of the properties in \Cref{thm:pullback}.
\end{proof}

Loosely speaking, the pullback by $T$ represents a change of variables from $(x,y)$ to $(r,v)$ by extending expanding the domain of $T$ to an invertible $T_{ij}(x,y)$ with the choice of $T_{ij}$ somewhat arbitrary.  
The uniqueness allows us to freely choose any other change of variables such that $T \circ T_{ij}(r,v) = r$, and the analysis on $T$ is still valid.
I.e., another valid choice of $T_{ij}$ is similar to polar-coordinates transformation $(T(x,y), \mathrm{Arg}(x,y))$.
Our choice was such that the analysis is straight-forward, although we will make use of the polar-coordinate variable transformation later to define the forward operator on linear representations.

The construction can be carried out much more generally for any smooth $T$ and is outlined in \cite{hormander1983}.
In this case, because of the specific form of $T$ under consideration, the induced pullback $T^\sharp$ is injective. 
This will be a consequence of the next lemma.
\begin{lem} \label{lem:innerProduct}
  For all $\rho\in\DD^*(\Omega_1)$ and $\psi \in \DD(\Omega_1)$, if we denote the ${h^{-1}}'$ as the derivative of the inverse of $h$ in \Cref{thm:pullback} 
  %and $\rho\cdot {h^{-1}}'$ as the distribution such that $\langle \rho \cdot {h^{-1}}',\psi\rangle = \langle \rho, {h^{-1}}' \cdot \psi\rangle$, 
  then
  \begin{equation}
    \langle T^\sharp \rho , \psi \circ T \rangle_{\Omega_2} = \pi \langle \rho,\psi \cdot {h^{-1}}'\rangle_{\Omega_1}.
  \end{equation}
\end{lem}
\begin{proof}
  First, note that both $\psi \circ T$ and ${h^{-1}}'\cdot \psi$ are elements of $\DD(\Omega_1)$.  
  From \eqref{eq:determinant}, observe
  \begin{align}
    \int_0^{\sqrt{h^{-1}(r)}} \left|dT_{ij}\right(r,t)| dt 
    &= \frac{{h^{-1}}'(r)}2 \int_0^{\sqrt{h^{-1}(r)}}\left(h^{-1}(r) - t^2\right)^{-1/2}\nonumber\\
    &= \frac\pi4 {h^{-1}}'(r).
  \end{align}
  Invoking \Cref{thm:testFunDensity}, let $\{\rho_n\}$ be a sequence in $\DD(\Omega_1)$ converging to $\rho$ in $\DD^*(\Omega_1)$, then substituting $\rho_n$ for $\rho$ and $\psi \circ T$ for $\phi$ in \eqref{eq:phiEquation}, we have
  \begin{align}
    \langle T^\sharp \rho_n, \psi \circ T \rangle_{\Omega_2}  &= 4\int_0^\infty \rho_n(r) \psi(r)  \left(\int_0^{\sqrt{h^{-1}(r)}} \left|dT_{ij}\right| dt\right) dr \nonumber \\
      &= \pi \int_0^\infty \rho_n(r) \psi(r)\,{h^{-1}}'(r) dr \nonumber\\
      &= \pi \left\langle\rho_n, \psi  \cdot {h^{-1}}' \right\rangle_{\Omega_1}. \label{pullbackinnerprod}
  \end{align}
  By continuity of $T^\sharp$, the desired equality follows.
\end{proof}

\begin{cor} \label{cor:injectivity}
  The map $T^\sharp:\DD^*(\Omega_1) \to \DD^*(\Omega_2)$ is injective.
\end{cor}
\begin{proof}
  Suppose $\rho \in \DD^*(\Omega_1)$ is such that $T^\sharp \rho = 0$.
  %and let $\{\rho_n\}\subset \DD(\Omega_1)$ converge to $\rho$ in $\DD^*(\Omega)$. For any $\psi \in \DD(\Omega_1)$,
  By the inverse function theorem, ${h^{-1}}'(r) = \frac 1{h'(r)} > 0$ since $h$ is increasing.
  So, $h' \cdot {h^{-1}}' (r) = 1$.
  %Let $\tilde h(r) = \frac 1{ {h^{-1}}'(r) }$, so $\tilde h \cdot {h^{-1}}' \equiv 1$.
  By \Cref{lem:innerProduct}, for an aribitrary $\psi\in \DD^(\Omega_1)$,
  \begin{equation}
    0 = \Big\langle T^\sharp \rho, (h' \cdot \psi) \circ T)\Big\rangle_{\Omega_2} = \Big\langle \rho, \psi\Big\rangle_{\Omega_1},
  \end{equation}
  thus, $T^\sharp$ has trivial kernel and as a linear map is injective.

%  Then, the product $\rho \cdot {h^{-1}}' \equiv 0$ and since ${h^{-1}}'$ has full support in $(0,\infty)$, $\rho \equiv 0$.  Invoking the density of $\DD(\Omega_1)$ from Theorem \ref{test_fun_density}, we have that $T^\sharp$ is an injection.
\end{proof}

%%%%%%%%%%%%%%%%%%%%%%%%%%%%%%%%%%%%%%%%%%%%%%%%%%%%%%%%%%%%%%%%%%%%%%%%%%%%%%%%
%                                                                              % 
%                              radially_symmetric_space                        % 
%                                                                              % 
%%%%%%%%%%%%%%%%%%%%%%%%%%%%%%%%%%%%%%%%%%%%%%%%%%%%%%%%%%%%%%%%%%%%%%%%%%%%%%%%
\subsection{The Space of Radially Symmetric \textcolor{red}{Functions}} \label{radially_symmetric_space}

\textcolor{red}{What follows in this section are my thoughts as formally as I can state them for defining $\KK\subset L^2(\Omega_2)$, the space of radially symmetric functions, and $\PP$ the space of their radial representations.  
I have run into trouble showing the $\KK$ is closed in $L^2(\Omega_2)$.
I don't intend for this to be in the paper in this form, but I am trying to present it in the form that seems most reasonable for solving the problem.
The gist is that I now believe that an extra condition is needed on $h$ so that $\KK$ is closed, but I am not sure what it should be.
Alternatively, maybe I should change $\KK$ or $\PP$. 
}

We will want to only consider radial representations, $\PP$, that correspond to square integrable radially symmetric functions, $\KK$.
Therefore, a reasonable space for radial representations is $\PP = T^{\sharp-1}(L^2(\Omega_2))$. 
If we restrict $T^{\sharp}$ to $\PP$, then the image of this map is $T^{\sharp} (T^{\sharp-1}(L^2(\Omega_2)))$, and it is a straight-forward set theory argument to show that this is precisely $\KK = T^\sharp(\DD^*(\Omega_1) \bigcap L^2(\Omega_2)$. 
Here is the argument:
\begin{prop}
  Let $f:X\to Y$ with $B \subseteq Y$, then $f (f^{-1} ( B )) = f(X) \bigcap B$. 
\end{prop}
\begin{proof}
Let $y \in f( f^{-1}(B))$, then there exists a $x \in f^{-1}(B)$ such that $f(x) = y$.  
Since $x \in f^{-1}(B)$, this implies that $f(x) = y \in B$. 
We have shown $y\in f(X)$ and $y \in B$, hence $f(f^{-1}(B)) \subseteq F(X)\bigcap B$.

On the other hand, suppose $y \in f(X)$ and $y \in B$.
Then there exists $x\in X$ such that $y = f(x) \in B$.  Hence $x \in f^{-1}(B)$.  This implies $y = f(x) \in f(f^{-1}(B)$.
\end{proof}

Both $\PP$ and $\KK$ seem to be the natural choices for these spaces.  
Moreover, the map
\begin{equation}
  T^\sharp|_{\PP}:\PP \to \KK
\end{equation}
is a surjection.
Since $\KK \subset L^2(\Omega)$, we can endow it with the subspace topology, and endow $\PP$ with the topology whose open sets are of the form $T^{\sharp-1}(U)$.  
This makes $T^\sharp$ automatically continuous, and the correspondence of the open sets is given by \Cref{lem:innerProduct}

I have been struggling to prove that $\KK$ is closed in $L^2(\Omega_2)$, and I believe that it might not be true in general for all increasing diffeomorphisms $h:\Omega_1 \to \Omega_1$. 

One reason I believe $h$ is not restrictive enough is %that $\R = T^{\sharp-1}(L^2(\Omega_2)) \not\subseteq L^2(\Omega_1)$ for certain $h$.
the following sketch of a construction:
Consider the sequence $\{\rho_n\}\subset \DD(\Omega_1)$ given by the convolution
\begin{equation}
  \rho_n(r) = r^{-1/3}\chi_{r\ge1}(r) * \delta_n(r)
\end{equation}
where $\delta_n \in \DD(\Omega_1)$ is an approximate identity.
This sequence converges to a distribution $\rho \in L^2(\Omega_1)$.
Yet, if $h(t) = t^2$, then ${h^{-1}}'(r) = \frac 12 r^{-1/2}$, then invoking \Cref{lem:innerProduct},
\begin{align}
  \int_{\Omega_2} T^\sharp \rho_n \cdot \bar{T^\sharp \rho_n}\,dxdy
  &= \Big\langle T^\sharp \rho_n, \bar{\rho_n} \circ T\Big\rangle_{\Omega_2}\\
  &= \pi\Big\langle \rho_n, {h^{-1}}'\cdot \bar{\rho_n} \Big\rangle_{\Omega_1}\\
  &= \pi\int_{\Omega_1} \rho_n(r)^2 {h^{-1}}'(r) dr
%  &= \frac{\pi}{2}\int_0^\infty r^{-7/6}*\delta_n^2(r), dr\\
\end{align}
which defines a sequence diverging like $\int_1^\infty r^{-7/6} dr$. 
Hence $T^\sharp$ in this case is not continuous on $L^2(\Omega_1)$.
Observe, however, that the choice of $h(t)$ has the wrong concavity to make points more dense near the origin.  
I think I can reconcile this with $h$ of the form $h(t) = t^{a}$ where $0<a<1/2$, or maybe even $0<a<1$, but is considering only such functions too restrictive?
Although this doesn't say anything about whether or not $\KK$ is closed in $L^2(\Omega_2)$, it precludes the following sketch of an argument that I initially wanted to use:

Ideally, I would like to have $h$ so that I can prove that $\R \subseteq L^2(\Omega_1)$ and $T|_{L^2(\Omega_1)}$ continuously maps into $L^2(\Omega_2)$ (with respect to $L^2$ in both domain and range). 
Then, $\R = T^{\sharp-1}(L^2(\Omega_2))$ is closed in $L^2(\Omega_1)$.  
Moreover, by the open mapping theorem [Rudin pg. 48], $T^\sharp|_{\R}$ is an open map ($\R$ is a Hilbert space, thus an $F$-space).  Since $T^\sharp|_{\R}$ is injective, it is also a closed map (Use $T^\sharp(U^c) = T^\sharp(U)^c$). Thus $\KK = T^\sharp(\R)$ is closed in $L^2(\Omega_2)$.  
It would also be nice to show that the topology induced by $T^\sharp|_\R$ is coarser than that of $L^2(\Omega_1)$ so that convergence in the first implies convergence in the latter.

I have other ideas about proving this directly by considering limit points of $\KK$, but the above approach seems nice.
As you suspected, this is much more complicated that I initially anticipated.
Any comments, thoughts, suggestions as always are appreciated.

%, however, for $h(t) = t^2$, 

%With $T^\sharp$ well defined, we make the following definition.
%
%\begin{defn}
%  A distribution $f \in \DD^*(\Omega_2)$ has \emph{radial symmetry} if there exists a distribution $\rho \in \DD^*(\Omega_1)$ so that $f = T^\sharp \rho$. 
%  We denote the space of radially symmetric distributions in $\mathscr H^m$ with $\KK^m = T^\sharp(\DD^*(\Omega_1)) \bigcap \mathscr H^m(\Omega_2)$, and reserve $\KK$ for radially symmetric distributions in $L^2(\Omega_2)$.
%\end{defn}
%%A natural continuation from \eqref{pullbackinnerprod} in the proof of Lemma \ref{lem:injective} gives
%\begin{cor}
%  $\KK^m$ is closed in $\mathscr H^m(\Omega_2)$.\label{closure}
%\end{cor}
%\begin{proof}
%  Let $f_n$ be a sequence of test functions in $\KK^m$ converging to $f\in \mathscr H^m$ with respect the $\mathscr H^m$ norm. 
%  For each $f_n$, there exists a unique $\rho_n$ so that $f_n = T^\sharp \rho_n$.  
%  Since $f_n$ converges for $\alpha = 0$, we have that $\langle f_n, \phi\rangle$ is a Cauchy sequence for any $\phi \in \DD(\Omega_2)$, so for any $\psi \in \DD(\Omega_1)$, \eqref{pullbackinnerprod} implies
%  \begin{align}
%    \left\langle f_n - f_m, (1/{h^{-1}}')\cdot \psi\circ T\right\rangle = \left\langle T^\sharp (\rho_n-\rho_m) ,\, (1/{h^{-1}}') \cdot\psi \circ T\right\rangle_{\Omega_2}  
%	&= \pi \left\langle\rho_n - \rho_m, \psi \right\rangle_{\Omega_1}. 
%  \end{align} 
%  %By putting conjugate bars on the second arguments, we have that $\rho_n$ is Cauchy in $L^2(\Omega_1)$, so let $\rho$ be the weak limit of $\rho_n$.
%  Hence we can define $\langle \rho,\psi \rangle= \lim \langle \rho_n,\psi\rangle$.
%  Continuity of $T^\sharp$, shows that $f = T^\sharp \rho$, hence $f \in T^\sharp(D^*(\Omega_1))$, and $\mathscr H^m(\Omega_2)$ closed implies $f \in \KK^m$.
%\end{proof}
%\begin{defn}
%  If $\rho$ is such that $f = T^\sharp \rho$ for some $f \in \KK$, then we say $\rho$ is the \emph{radial representation} of $f$ and we denote the space $\R^m = {T^\sharp}^{-1}(\KK^m)$.
%\end{defn}
%%Since $\KK$ is closed, continuity of $T^\sharp$ implies $\R$ is closed.
%%Moreover, the construction of $\rho$ in Corollary \ref{closure} shows that $\R^m \subset \mathscr H^m(\Omega_1)$.
%%
%%To summarize, we have established a closed subspace of radially symmetric distributions $\KK \subset L^2(\Omega_2)$ and a one-to-one correspondence with their radial representations  $\R \subset L^2(\Omega_1)$. 
%%%In fact, using \cref{pullbackinnerprod}, we see that $\KK$ induces a structure on $\R$ by defining an inner product
%%Moreover, $\R$ inherits structure from $\KK$ as follows
%\begin{thm} \label{thm:isometry}
%  The restriction of $T^\sharp$ to $\R^m$ provides an isometry between $\KK^m$ with the standard energy inner products and $\R^m$ with
%\begin{equation}
%  (\rho_1,\rho_2)_{\R} := \left( T^\sharp \rho_1,\, T^\sharp \rho_2 \right)_{L^2(\Omega_2)} = \pi \left( \rho_1 \cdot {h^{-1}}',\, \rho_2 \right)_{L^2(\Omega_1)}. \label{eq:isometry}
%\end{equation}
%%\Cref{isometry} establishes an isometry between $\KK$ and $\R$, so all subspaces of $\KK$ correspond to subspaces of $\R$.
%%The subspace structure of $\mathscr K^m$ also induces a subspace structure on $\R^m$
%and
%\begin{equation}
%  (\rho_1,\rho_2)_{\R^m} := \sum_{|\alpha|\le m} \left( \del^\alpha T^\sharp \rho_1,\, \del^\alpha T^\sharp \rho_2 \right)_{\Omega_2}.
%\end{equation}
%\end{thm}
%\begin{proof}
%\end{proof}
%%however, explicit representations must be computed using the appropriate chain and product rules.

\end{chapter}
