\setlength{\parindent}{2ex}
%\newcommand{\mathscr}{\mathcal}
\begin{chapter}{Reconstruction on the Continuum}\label{chapter:theoretical}

The goal of this chapter is to develop the necessary mathematical tools to encapsulate prior notions of both smoothness and radial symmetry within the structure of a separable Hilbert space.
This is done within the framework of distributions from linear PDE theory. 

\begin{com}
  There is enough structure here to do what we want and maintain the structure of a Hilbert space. 
  This will make rigorous the discussion at the end of the last chapter for showing that the problem is ill-posed.
  It will also allow us to give rigorous formal definition for an infinite dimensional stochastic formulation of the problem.
  This will be the basis for estimation in the following chapter.
\end{com}

\section{Distribution spaces}
In this section we establish the preliminary definitions and main results from distribution theory.
There are several treatments of the subject in varying levels of generality, and this work draws primarily from \citep{richtmyer1978principles,hormander1983,rudin1991,griffel2002,strichartz2003guide}.

  \subsection{The space of test functions and distributions}

\begin{com}
The space of distributions provides  blah blah, a bunch of bullshit from PDEs and incidentally integral equations.
\end{com}

Let $\DD(\Omega)$ be the space of compactly supported smooth functions defined on an open set $\Omega \subseteq \R^k$.
Endow $\DD(\Omega)$ with the topology such that $(\phi_n)\subset \DD(\Omega)$ converges there exists a compact set $K$ such that 
\begin{equation}
  \bigcup_{n=1}^\infty \supp \phi_n \subseteq K \quad\text{and}\quad
  \sup_{m\ge n} \left|\del^\alpha (\phi_n-\phi_m)\right|\to 0, \label{eq:testFunctionContinuity}
\end{equation} 
for any multi-index such that $|\alpha|\le k$.
That is,  $\alpha$ is a $k$-tuple of non-negative integers $(\alpha_1\dots \alpha_k)$, such that $\sum \alpha_i \le k$ and $\del^\alpha = \prod \left(\frac{\del}{\del x_i}\right)^{\alpha_i}$.
In distribution theory, these are called \emph{test functions} on $\Omega$.
The space of continuous linear functionals, denoted $D^*(\Omega)$, are the \emph{distributions} on $\Omega$.
We adopt the notation $\langle f, \phi\rangle$ for the action of a linear functional $f$ on $\phi \in \DD(\Omega)$ and freely use the natural inclusion of functions $g \to \tilde g \in \DD^*(\Omega)$ by $\langle \tilde g, \phi \rangle = \int g\, \phi\,dx$ when the integration exists and omit the tilde notation distinguishing $g$ and $\tilde g$ as the representation should be clear from context. 

We state two general topological results regarding distributions. See \citep[Chapter 2]{hormander1983} for the proofs of each.
\begin{thm} \label{thm:completeness}
  Suppose $(f_n)$ is a sequence in $\DD^*(\Omega)$ such that $\lim\langle f_n,\phi\rangle$ exists for all $\phi \in \DD(\Omega)$, then there exists a unique $f\in \DD^*(\Omega)$ such that
  \begin{equation}
    \langle f, \phi\rangle = \lim_{n\to\infty} \langle f_n, \phi \rangle.
  \end{equation}
\end{thm}
The existence of such of the bilinear form $f$ can be readily established by using the completeness of the associated field, either $\R$ or $\C$, and the main difficulty of establishing the result is showing that the resulting linear functional is continuous on $\DD(\Omega)$.

The next result, sometimes referred to as localization, establishes a dense embedding of the $\DD(\Omega)$ into $\DD^*(\Omega)$.
\begin{thm} \label{thm:density}
  Given $f\in \DD^*(\Omega)$, there exists a sequence $(\phi_n) \subset \DD(\Omega)$ such that 
  \begin{equation}
    \langle f,\psi\rangle = \lim_{n\to\infty}\langle \phi_n,\psi\rangle.
  \end{equation}
\end{thm}

These results allow for operators defined on $\DD(\Omega)$ to be extended in a continuous way to $\DD^*(\Omega)$ so long as one can define an adjoint operation with respect to the bilinear form $\langle \cdot,\cdot \rangle$.
The classical example is extending the differential operator $\frac{\del}{\del x_i}:\DD^*(\Omega) \to \DD^*(\Omega)$. 
First, for test functions observe that integrating by parts and using the compactness of the support of $\psi$ yields
\begin{equation}
  \left\langle \frac{\del}{\del x_i} \phi,\psi \right\rangle = \int_{\Omega} \frac{\del}{\del x_i} \phi(x) \psi(x) dx = - \int_{\Omega}\phi(x) \frac{\del}{\del x_i} \psi(x)dx =- \left\langle \phi, \frac{\del}{\del x_i} \psi\right\rangle.
\end{equation}
This motivates defining $\frac{\del}{\del x_i} f \in \DD^*(\Omega)$ by
\begin{equation}
  \left\langle \frac{\del}{\del x_i} f, \phi \right\rangle 
    \eqdef  - \left\langle f, \frac{\del}{\del x_i} \psi\right\rangle
\end{equation}
from which we can extend the definition of $\del^\alpha$
\begin{equation}\label{eq:distributionalDerivative}
  \left\langle \del^\alpha f, \phi \right\rangle 
    \eqdef  (-1)^{|\alpha|} \left\langle f, \del^\alpha \psi\right\rangle.
\end{equation}
For a given sequence $(\phi_n)\subset \DD(\Omega)$ converges uniformly in all derivatives, the functional defined in \eqref{eq:distributionalDerivative} is continuous, hence is in $\DD^*(\Omega)$.
Since the operator $\del^\alpha:\DD^*(\Omega)\to\DD^*(\Omega)$ is expressed as an adjoint on test functions with respect to evaluation, its continuity follows directly from the weak-topology induced from $\DD(\Omega)$, i.e., suppose $f_n \to f$ in $\DD^*(\Omega)$, then
\begin{align}
  \lim_{n\to \infty}\langle \del^{\alpha}f_n, \phi\rangle 
  = \lim_{n\to\infty}(-1)^{|\alpha|} \langle f_n,\del^{\alpha}\phi\rangle
  = (-1)^{|\alpha|} \langle f,\del^{\alpha}\phi\rangle
  = \langle \del^{\alpha}f, \phi\rangle.
\end{align}

This approach serves as a template for how we will extend the radial change of variables in \Cref{chapter:introduction} to larger spaces of distributions.

\subsection{$L^2$ as a subspace of distributions}

\begin{com}
  We need metrics and norms and algebra to play nice together.
  The best space for that is a Hilbert space, and in particular a closed substace of $L^2$.
  Here we show how to view the space of $L^2$ functions as a subspace of $\DD^*$.
\end{com}
The development follows \citep{richtmyer1978principles}, for which we need only the notion the $L^2$ inner product, as opposed to Fourier based approaches which can be found in \citep{rudin1991,hormander1983}.
This development provides several details that are omitted in \citep{richtmyer1978principles}. 

We define the $L^2$ inner-product for test functions as the bilinear form $(\cdot,\cdot)_{L^2(\Omega)}:\DD(\Omega)\times\DD(\Omega) \to \C$ by 
\begin{equation} \label{eq:l2innerProduct}
  (\phi,\psi)_{L^2(\Omega)} \eqdef \int_{\Omega} \phi(x)\bar{\psi(x)}\,dx,
\end{equation}
with the induced norm 
\begin{equation} \label{eq:l2norm}
  \|\phi\|^2_{L^2(\Omega)} \eqdef (\phi,\phi)_{L^2(\Omega)}.
\end{equation}
The linearity of \eqref{eq:l2innerProduct} is inherited by the linearity of integration and positivity follows from the positivity of $\phi(x)\bar{\phi(x)} = |\phi(x)|^2$. 
For definiteness, note that if $\phi = 0$ then $\|\phi\|_{L^2(\Omega)} = 0$, and only if $\phi = 0$, otherwise, continuity of $\phi$ implies a neighborhood where $|\phi(x)| > 0$ which gives $\|\phi\|_{L^2(\Omega)} > 0$.
Since \eqref{eq:l2innerProduct} defines an inner-product, the triangle inequality of the norm follows from the Cauchy-Schwarz-Bunyakovsky inequality.

A sequence $(\phi_n)\subset \DD(\Omega)$ is Cauchy with respect to $L^2(\Omega)$ if
\begin{equation} 
  \sup_{k\ge n} \|\phi_n - \phi_k\|_{L^2(\Omega)} \to 0.
\end{equation}
Observe that $(\phi,\psi)_{L^2(\Omega)} = \langle \phi,\bar{\psi}\rangle$ when $\phi$ is viewed as an element of $\DD^*(\Omega)$.
Hence, if $(\phi_n)$ is Cauchy with respect to $L^2(\Omega)$, then the sequence $\big(\langle \phi_n,\psi\rangle\big)$ is a Cauchy sequence of complex numbers, i.e., using Cauchy-Schwarz-Bunyakovsky
\begin{equation}
  |\langle \phi_n, \psi \rangle - \langle \phi_k,\psi\rangle| 
    = |(\phi_n - \phi_k,\bar{\psi})_{L^2(\Omega)}|  
    \le \|\phi_n - \phi_k\|_{L^2(\Omega)}\|\psi\|_{L^2(\Omega)}. \label{eq:cauchySchwarz}
\end{equation}
Hence, $\lim\langle \phi_n,\psi\rangle$ exists for all $\psi \in \DD(\Omega)$ (by completeness of $\C$), and \Cref{thm:completeness} provides uniquely an $f \in \DD^*(\Omega)$ such that
\begin{equation}
  \lim_{n\to\infty} \langle \phi_n, \psi\rangle = \langle f,\psi\rangle.
\end{equation}
All such $f$ are elements of the space $L^2(\Omega)$.

Sequences $(\phi_n)$ and $(\psi_n)$ are \emph{equivalent} $L^2(\Omega)$ Cauchy sequences if
\begin{equation}\label{eq:cauchyEquivalence}
  \|\phi_n - \phi_n'\|_{L^2(\Omega)} \to 0.
\end{equation}
These distributions are well defined in the following sense: 
\begin{prop} \label{prop:cauchyCorrespondence}
  Sequences $(\phi_n)$ and $(\phi_n')$ determine the same distribution if and only if they are equivalent.
\end{prop}
\begin{proof}
  Suppose $(\phi_n)$ and $(\phi_n')$ are equivalent Cauchy sequences for $f$ and $f'$ respectively.
  Arguing similarly to \eqref{eq:cauchySchwarz},
  \begin{align} 
    |\langle f, \psi\rangle - \langle g, \psi\rangle| = \left|\lim_{n\to \infty}\langle \phi_n - \phi_n', \psi\rangle\right| \le \lim_{n\to\infty}\|\phi_n - \phi_n'\|_{L^2(\Omega)}\|\psi\|_{L^2(\Omega)} =  0
  \end{align} 
  for all $\psi \in \DD^(\Omega)$, hence $f = f'$.

  Conversely, suppose $(\phi_n)$ and $(\phi_n')$ are Cauchy sequences that determine the same distribution $f$, hence
  \begin{equation}
    \lim_{n\to\infty} \langle \phi_n,\psi\rangle = \lim_{n\to\infty} \langle \phi_n',\psi\rangle = \langle f,\psi\rangle
  \end{equation}
  for all $\psi \in \DD^*(\Omega)$.
  The sequence given by $\xi_n = \phi_n - \phi_n'$ is also Cauchy by the triangle inequality, and 
  \begin{equation}
    \lim_{n\to\infty}\langle \xi_n,\bar{\psi}\rangle^2 = \lim_{n\to\infty} (\xi_n,\psi)_{L^2(\Omega)}^2 = 0.
  \end{equation}
  Let $\eps >0$ be given and $n_0$ sufficiently large such that 
  \begin{equation}
    \sup_{k\ge n_0} \|\xi_{n_0} - \xi_k\|_{L^2(\Omega)}^2 < \eps.
  \end{equation}
  Using Cauchy-Schwarz-Bunyakovsky,
  \begin{align}
    \|\xi_n\|_{L^2(\Omega)}^2 
      &= |(\xi_n,\xi_n)_{L^2(\Omega)}| \nonumber\\
      &= |(\xi_n,\xi_n - \xi_{n_0})_{L^2(\Omega)}| \nonumber\\
      &\le \|\xi_n\|_{L^2(\Omega)}\|\xi_n-\xi_{n_0}\|_{L^2(\Omega)}
  \end{align}
  we have that for all $\|\xi_n\|_{L^2(\Omega)} \not= 0$
  \begin{equation}
    \|\xi_n\|_{L^2(\Omega)} \le \|\xi_n - \xi_{n_0}\|_{L^2(\Omega)} < \eps
  \end{equation}
  for all $n \ge n_0$, hence $\|\xi_n\|_{L^2(\Omega)} \to 0$ which implies $\phi_n$ and $\phi_n'$ are equivalent Cauchy sequences.
\end{proof}

It can be readily shown that \eqref{eq:cauchyEquivalence} defines an equivalence relation on Cauchy sequences for which vector addition and scalar multiplication are well-defined, hence, the equivalence classes of Cauchy sequences form linear subspace of $\DD^*(\Omega)$.

We can now extend the inner product to elements in $L^2(\Omega)$ in the following way,
\begin{equation} \label{eq:l2innerProductFull}
  (f,g)_{L^2(\Omega)} \eqdef \lim_{n\to\infty} (\phi_n,\psi_n)_{L^2(\Omega)}
\end{equation}
where $(\phi_n)$ and $(\psi_n)$ are Cauchy sequences corresponding to $f$ and $g$ respectively.
\begin{prop}
  The limit in \eqref{eq:l2innerProductFull} exists and is well-defined for equivalent Cauchy sequences. 
  Moreover, $\lim_{n\to\infty} \langle f,\bar{\psi_n}\rangle _{L^2(\Omega)} = (f,g)_{L^2(\Omega)}$.
\end{prop}
\begin{proof}
  Since $(\phi_n)$ and $(\psi_n)$ are Cauchy, the inequality $|\|\phi_n\|_{L^2(\Omega)} - \|\phi_k\|_{L^2(\Omega)}| \le \|\phi_n - \phi_k\|_{L^2(\Omega)}$ implies that $\|\phi_n\|_{L^2(\Omega)}$ and $\|\psi_n\|$ are both Cauchy sequences of positive numbers, and thus have finite limits.

Now observe,
\begin{align}
  |(\phi_n,\psi_n)_{L^2(\Omega)} - (\phi_k,\psi_k)_{L^2(\Omega)}| 
    &= (\phi_n,\psi_n-\psi_k)_{L^2(\Omega)} -(\phi_n-\phi_k,\psi_k)_{L^2(\Omega)}| \nonumber\\
    &\le \|\phi_n\|_{L^2(\Omega)}\|\psi_n-\psi_k\|_{L^2(\Omega)} + \|\phi_n - \phi_k\|_{L^2(\Omega)}\|\psi_k\|_{L^2(\Omega)} \nonumber\\
    &\le \|\phi_n\|_{L^2(\Omega)}\|\psi_n-\psi_k\|_{L^2(\Omega)} \nonumber\\
    &\quad\quad+ \|\phi_n - \phi_k\|_{L^2(\Omega)}\big(\|\psi_k - \psi_n\|_{L^2(\Omega)} + \|\psi_n\|_{L^2(\Omega)}\big)
\end{align}
and taking $\sup_{k\ge n}$ then limiting in $n$ on the right hand side of the inequality results in 0 since $(\|\phi_n\|)$ and $(\|\psi_n\|$ have finite limits.
Thus, $(\phi_n,\psi_n)$ is a Cauchy sequence in $\C$, and hence, has a finite limit.

Let $\eps > 0$ be given. 
For all $n$, choose $m$ sufficiently large so that
\begin{align}
  |\langle f,\bar{\psi_n}\rangle - (f,g)_{L^2(\Omega)}| 
    &\le |\langle f - \phi_m,\bar{\psi_n}\rangle| + |\langle \phi_m - \phi_n, \bar{\psi_n}\rangle| + |\langle \phi_n,\bar{\psi_n}\rangle - (f,g)_{L^2(\Omega)}| \nonumber\\
    &\le |\langle f - \phi_m,\bar{\psi_n}\rangle| + \| \phi_m - \phi_n\| \|\psi_n\| + |\langle \phi_n,\bar{\psi_n}\rangle - (f,g)_{L^2(\Omega)}| \nonumber\\
    &< \eps +|\langle \phi_n,\bar{\psi_n}\rangle - (f,g)_{L^2(\Omega)}|.
\end{align}
Taking limits on both sides gives 
\begin{equation} \label{eq:iteratedInnerProduct}
  \lim_{n\to\infty} \langle f,\bar{\psi_n}\rangle = (f,g)_{L^2(\Omega)}, 
\end{equation}
since $\eps>0$ is arbitrary.

Finally, to show that the inner product is well-defined, suppose $(\phi_n')$ and $(\psi_n')$ are equivalent Cauchy sequences to $(\phi_n)$ and $(\psi_n)$ respectively.
\begin{align}
%  |(f,g) - (f',g')| \lim_{n\to\infty}
%  |(\phi_n,\psi_n)_{L^2(\Omega)} - (\phi_n',\psi_n')_{L^2(\Omega)}|
%  &=|(\phi_n,\psi_n - \psi_n')_{L^2(\Omega)} - (\phi_n'-\phi_n,\psi_n')_{L^2(\Omega)}| \nonumber \\
%  &=|\langle\bar{\psi_n - \psi_n'},\phi_n\rangle - \langle\phi_n'-\phi_n,\bar{\psi_n'}\rangle| \label{eq:innerProductWellDefined}
%  = |(\phi_n - \phi_n',\phi_n - \phi_n') + (\psi_n - \psi_n',\psi_n - \psi_n') - (\phi_n',\psi_n')|
  |\lim_{n\to\infty}(\phi_n,\psi_n)_{L^2(\Omega)} - (\phi_n',\psi_n')_{L^2(\Omega)}| 
  &= |\lim_{n\to\infty}\langle f,\bar{\psi_n}\rangle - \lim_{n\to\infty}\langle f',\bar{\psi_n}'\rangle| &\text{by \eqref{eq:iteratedInnerProduct}}\nonumber\\
  &= |\lim_{n\to\infty}\langle f,\bar{\psi_n} - \bar{\psi_n}'\rangle| &\text{by \Cref{prop:cauchyCorrespondence}}\nonumber \\
  &= 0 \nonumber.
\end{align}
Thus, the inner product is well-defined for equivalent Cauchy sequences.
\end{proof}

Showing that \eqref{eq:l2innerProductFull} is an inner product on $L^2(\Omega)$ is straight-forward, and the resulting inner product space is a Hilbert space, as stated in the following theorem.
\begin{thm}
The space $L^2(\Omega)$ is complete complete with respect to the inner product defined in \eqref{eq:l2innerProductFull}, hence, is a Hilbert space.
\end{thm}
%\begin{proof}
%  Let $(f_n)\subseteq $ be a Cauchy sequence with respect to \eqref{eq:l2innerProductFull}. 
%\end{proof}
See \citep{richtmyer1978principles} for the proof, which follows a standard diagonalization argument. %similar to the Arzela-Ascoli theorem.
We remark that there is a correspondence with the standard notion of $L^2(\Omega)$ with respect to Lebesgue measure and this development.
The basis of correspondence comes from the result that simple function can be arbitrarily approximated with a sequence of test functions in the standard $L^2(\Omega)$ sense with Lebesgue measure. 
See \citep{hormander1983} for a rigorous development of this correspondence.
%\begin{com}
%  Use the hard work in \Cref{thm:density} maybe.
%\end{com}
\subsection{The Sobolev spaces $\HH^n(\Omega)$ and $\HH_0^n(\Omega)$}
Sobolev spaces provide a framework for imposing regularity on distributions in terms of their derivatives.
In this subsection, we briefly overview the definition of these spaces, and state a version of the Sobolev embedding theorem sufficient for characterizing PSFs of interest in this work.

\section{Radial symmetry}
Symmetry is established by casting it in terms of a many-to-one smooth map $T$ that is constant on circles of a fixed radius.  
If $f$ is a function on $\R^2$ and there exists a function $\rho$ so that $f(x,y) = \rho(\sqrt{x^2 + y^2})$, then observe that $f$ has the common notion of radial symmetry with a radial profile $\rho$.
This notion is easily adapted to distributions by developing the corresponding linear pullback operator to $T^\sharp$ that maps $\rho$ to $f$ by precomposition by $T(x,y)$ on sequences of test functions converging to $\rho$.

\subsection{The pull-back operator}
In this subsection, we will explicitly construct the pullback operator $T^\sharp$ for a slightly more general smooth map $T(x,y)$. 
It will turn out that our choice of $T$ will induce a pullback operator that is injective.
This is the content of the following theorem.

This is the content of the following theorem.
\begin{thm} \label{thm:pullback}
  Let $\Omega_1 := (0,\infty) \subset \R$ and $\Omega_2 := \R^2 \setminus \{x = 0\text{ or }y=0\}$.
  For $h:\Omega_1\to\Omega_1$ such that $h(t) = t^a$ for $0<a<1$, let $T:\Omega_2 \to \Omega_1$ by $T(x,y) = h(x^2 + y^2)$, then there exists a unique, injective, continuous, linear operator $T^\sharp:\DD^*(\Omega_1) \to \DD^*(\Omega_2)$ so that $\langle T^\sharp \rho ,\phi\rangle = \langle \rho \circ T,\phi\rangle$ for all $\phi \in \DD(\Omega_2)$ and $\rho \in \DD(\Omega_1)$.
\end{thm}

To prove this result, we first establish the following lemmas.

\begin{lem} \label{lem:existence}
  There exists a map $T_{\sharp}:\DD(\Omega_2) \to \DD(\Omega_1)$ so that for any $\rho \in \DD(\Omega_1)$
  \begin{equation}
    \langle \rho \circ T, \phi \rangle_{\Omega_2} = \langle \rho, T_\sharp \phi\rangle_{\Omega_1}.
  \end{equation}
\end{lem}
\begin{proof}
  Let $Q_{ij} = \{ (x,y): (-1)^i x>0, (-1)^jy>0\}$ for $i,j \in \{0,1\}$ so that $\bigcup Q_{ij} = \Omega_2$.
  Define $T_{ij}:Q_{ij} \to R\subset \R^2$ by 
  \begin{equation}
    T_{ij}(x,y) = \Big(T(x,y), (-1)^jy\Big). 
  \end{equation}
  Observe that each $T_{ij}$ is a diffeomorphism onto $R = \left\{(r,t): 0 < t < \sqrt{h^{-1}(r)}\right\} = \left\{(r,t): 0 < t < r^{\frac{1}{2a}}\right\}$ with inverse 
  \begin{align}
    T_{ij}^{-1}(r,t) 
      &= \Big((-1)^i\sqrt{h^{-1}(r) - t^2}, (-1)^jt\Big),\nonumber\\
      &= \Big((-1)^i\sqrt{r^{\frac 1a} - t^2}, (-1)^jt\Big),\\
    \intertext{ and }
    \left|dT_{ij}^{-1}(r,t)\right| 
      &= \frac12 \frac{\del}{\del r}[h^{-1}(r)]\left(h^{-1}(r) - t^2\right)^{-1/2}\nonumber \\
      &= \frac{1}{2a} r^{\frac 1a-1}\left(r^{\frac 1a} - t^2\right)^{-1/2}, \label{eq:determinant}
  \end{align}
  which is positive and smooth for all $(r,t)\in \Omega_2$.
  Furthermore, note that 
  \begin{equation}
    T \circ T_{ij}^{-1}(r,t) = r. \label{eq:partialInverseT}
  \end{equation}
  Now, given $\rho \in \DD(\Omega_1)$, a change of variables results in
  \begin{align}
    \langle \rho \circ T, \phi\rangle_{\Omega_2} 
%    &= \iint_{\Omega_2} \rho\circ T(x,y) \phi(x,y)dxdy \\
    &= \sum_{ij}\iint_{Q_{ij}} \rho\circ T(x,y)\cdot \phi(x,y)dxdy \nonumber \\
    &= \sum_{ij}\iint_{R} \rho(r)\cdot  \phi \circ T_{ij}^{-1}(r,t)\left|dT_{ij}\right|drdt \nonumber \\
    &= \int_0^\infty \rho(r) \left(\int_0^{\sqrt{h^{-1}(r)}} \sum_{ij}\phi \circ T_{ij}^{-1}(r,t)\left|dT_{ij}\right|dt \right)dr \label{eq:phiEquation}.
%    &=: \int_{\Omega_1} \rho(r) \Phi_{\phi,T_{ij}}(r) dr. 
%    &= \langle \rho, \Phi_{\phi,T_{ij}} \rangle_{\Omega_1},
  \end{align}
  Let 
  \begin{align}
    [T_\sharp\phi](r) 
      &= \int_0^{\sqrt{h^{-1}(r)}} \sum_{ij}\phi \circ T_{ij}^{-1}(r,t)\left|dT_{ij}\right|dt \\
      &= \frac{r^{\frac 1a -1} }{2a}\sum_{ij}\int_0^{r^{\frac{1}{2a}}} \phi \left((-1)^i\sqrt{r^{\frac 1a} - t^2}, (-1)^jt\right) \left( r^{\frac 1a} - t^2 \right)^{-1/2} dt, \label{eq:psiPhiDef}
  \end{align}
  and we must show that $T_\sharp\phi \in \DD(\Omega_1)$.
  Note that $\supp \left(\phi \circ T_{ij}^{-1} \right) = T_{ij}( \supp \phi )$ is compact in $R$ as the continuous image a compact set, and since $T_{ij}$ is a diffeomorphism, $\phi \circ T_{ij}^{-1} \in \DD(\Omega_2)$. 
%  Moreover, integrating marginally over $t$ and summing over ${i,j}$ are smooth operations, so $\Phi_{\phi,T_{ij}}$ is smooth. 
%  The support of $\Phi_{\phi,T_{ij}}$ is the projection of the support of $\phi\circ T_{ij}^{-1}$, hence is compact.
%  Marginially integrating $\phi\circ T_{ij}^{-1}$ results in a test function, thus $\Phi_{\phi,T_{ij}} \in \DD(\Omega_1)$. 
  Since $\phi\circ T_{ij}^{-1}(r,t)$ is smooth, a standard result from analysis \cite[pg. 433]{strichartz2000} implies that $\int \phi \circ T_{ij}^{-1}(r,t)\,dt$ is a smooth function in $r$.  
  The support of this function is the projection of the support of $\phi\circ T_{ij}$ onto the second coordinate, and hence, is compact.
  Summing over $i,j$ results results in a compactly supported smooth function.

\end{proof}

  Using \Cref{lem:existence}, define
  \begin{defn}
    Let $T^\sharp:\DD^*(\Omega_1) \to \DD^*(\Omega_2)$ by
  \begin{equation}
    \langle T^\sharp \rho, \phi \rangle_{\Omega_2} := \langle \rho, T_\sharp\phi\rangle_{\Omega_1}.
  \end{equation}
  \end{defn}
  To see that $T^\sharp \rho \in \DD^*(\Omega_2)$ (i.e. acts continuously on $\DD(\Omega_2)$ as a linear functioal), let $\{\phi_n\} \to 0$ in $\DD(\Omega_2)$, so in particular ($\alpha = 0$ in \eqref{eq:testFunctionContinuity}), $\sup|\phi_n| \to 0$. 
  Then, by \eqref{eq:psiPhiDef}, $\sup |T_\sharp\phi_n| \to 0$, and thus $\langle \rho, T_\sharp\phi_n\rangle \to 0$ by the continuity of $\rho$. 
%  Moreover, if we take $\rho = Id_{\DD(\Omega_1)}$ the iden, then $\langle Id_{}$
%  $T_\sharp:\DD(\Omega_2) \to \DD(\Omega_1)$.

  The linearity and continuity of $T^\sharp$ follow directly from this definition.
  That is 
  \begin{align}
    \langle T^\sharp \rho_1 + \alpha T^\sharp \rho_2,\phi\rangle_{\Omega_2} 
    &= \langle T^\sharp\rho_1,\phi\rangle_{\Omega_2} + \alpha\langle T^\sharp\rho_2 ,\phi \rangle_{\Omega_2} \nonumber\\ 
    &= \langle \rho_1,T_\sharp\phi\rangle_{\Omega_1} + \alpha\langle \rho_2 ,T_\sharp\phi \rangle_{\Omega_1} \nonumber\\
    &= \langle T^\sharp(\rho_1+\alpha \rho_2),\phi\rangle_{\Omega_2} \label{eq:linearity} 
  \end{align} 
  and if $\langle \rho_n, \psi\rangle \to 0$ for all $\psi \in \DD(\Omega_1)$, then 
  \begin{equation}
    \langle T^\sharp \rho_n,\phi\rangle_{\Omega_2} = \langle \rho_n,T_\sharp\phi\rangle_{\Omega_1} \to 0.
  \end{equation}

%As a corollary to this result, we can establish the existence in the statement of \Cref{thm:pullback}. That is, define $T^\sharp:\DD^*(\Omega_1) \to \DD^*(\Omega_2)$ by 
Loosely speaking, the pullback by $T$ represents a change of variables from $(x,y)$ to $(r,v)$ by expanding the domain of $T$ to an invertible $T_{ij}(x,y)$ with the choice of $T_{ij}$ somewhat arbitrary.  
For example, another valid choice of $T_{ij}$, which is similar to a polar-coordinates transformation, would be $(T(x,y), \mathrm{Arg}(x,y))$.
When uniqueness of $T^\sharp$ is shown, it will allow us to freely choose any other change of variables such that $T \circ T_{ij}(r,v) = r$, and the analysis on $T$ will still be valid.
Our choice was such that the analysis is straight-forward, although we will make use of the polar-coordinate variable transformation later to define the forward operator on linear representations.
The existence and continuity of a pullback operator can be carried out much more generally for any smooth $T$ and is outlined in \cite{hormander1983}.
However, in this case, because of the specific form of $T$ under consideration, the induced pullback $T^\sharp$ is injective. 
This will be a consequence of the next lemma.
\begin{lem} \label{lem:innerProduct}
  For all $\rho\in\DD^*(\Omega_1)$ and $\psi \in \DD(\Omega_1)$, then 
  then
  \begin{equation}
    \langle T^\sharp \rho , \psi \circ T \rangle_{\Omega_2} = \pi \langle \rho,\psi \cdot {h^{-1}}'\rangle_{\Omega_1}.
  \end{equation}
  where ${h^{-1}}'$ is the derivative of the inverse of $h$ in \Cref{thm:pullback}. 
\end{lem}
\begin{proof}
  First, note that both $\psi \circ T$ and ${h^{-1}}'\cdot \psi$ are elements of $\DD(\Omega_1)$.  
  From \eqref{eq:determinant}, observe
  \begin{align}
    \int_0^{\sqrt{h^{-1}(r)}} \left|dT_{ij}\right(r,t)| dt 
    &= \frac{{h^{-1}}'(r)}2 \int_0^{\sqrt{h^{-1}(r)}}\left(h^{-1}(r) - t^2\right)^{-1/2}\nonumber\\
    &= \frac\pi4 {h^{-1}}'(r).
  \end{align}
  Invoking \Cref{thm:testFunDensity}, let $\{\rho_n\}$ be a sequence in $\DD(\Omega_1)$ converging to $\rho$ in $\DD^*(\Omega_1)$, then substituting $\rho_n$ for $\rho$ and $\psi \circ T$ for $\phi$ in \eqref{eq:phiEquation}, we have
  \begin{align}
    \langle T^\sharp \rho_n, \psi \circ T \rangle_{\Omega_2}  &= 4\int_0^\infty \rho_n(r) \psi(r)  \left(\int_0^{\sqrt{h^{-1}(r)}} \left|dT_{ij}\right| dt\right) dr \nonumber \\
      &= \pi \int_0^\infty \rho_n(r) \psi(r)\,{h^{-1}}'(r) dr \nonumber\\
      &= \pi \left\langle\rho_n, \psi  \cdot {h^{-1}}' \right\rangle_{\Omega_1}. \label{eq:pullbackInnerProd}
  \end{align}
  By continuity of $T^\sharp$, the desired equality is established.
\end{proof}

We can now proceed to prove \Cref{thm:pullback}.
\begin{proof}
  It remains to show that 

  Uniqueness is a consequence of \Cref{thm:testFunDensity}. That is, suppose $T^\dagger:\DD^*(\Omega_1) \to \DD^*(\Omega_2)$ is a continuous linear functional such that $\langle T^\dagger \rho,\phi \rangle = \langle \rho \circ T, \phi\rangle$ for all $\phi \in \DD(\Omega_2)$ whenever $\rho \in \DD(\Omega_1)$. Then, for any $\rho \in \DD^*(\Omega_1)$, let $\{\rho_n\}\subset \DD(\Omega_1)$ converge to $\rho$ (in the $\DD^*(\Omega_1)$ sense), so 
  \begin{equation}
    \left\langle (T^\sharp - T^\dagger)\rho,\phi\right\rangle_{\Omega_2} = \lim \langle T^\sharp\rho_n,\phi\rangle_{\Omega_2} - \lim \langle T^\dagger\rho_n,\phi\rangle_{\Omega_2} = 0.
  \end{equation}
  Hence $T^\sharp = T^\dagger$.
  
  It remains to show that $T^\sharp$ is injective. Suppose $\rho \in \DD^*(\Omega_1)$ is such that $T^\sharp \rho = 0$.
  %and let $\{\rho_n\}\subset \DD(\Omega_1)$ converge to $\rho$ in $\DD^*(\Omega)$. For any $\psi \in \DD(\Omega_1)$,
  By the inverse function theorem,
  ${h^{-1}}'(r) = \frac 1{h'(r)} > 0$ since $h$ is increasing.
  So, $h' \cdot {h^{-1}}' (r) = 1$.
  %Let $\tilde h(r) = \frac 1{ {h^{-1}}'(r) }$, so $\tilde h \cdot {h^{-1}}' \equiv 1$.
  By \Cref{lem:innerProduct}, for an aribitrary $\psi\in \DD(\Omega_1)$,
  \begin{equation}
    0 = \Big\langle T^\sharp \rho, (h' \cdot \psi) \circ T)\Big\rangle_{\Omega_2} = \Big\langle \rho, \psi\Big\rangle_{\Omega_1},
  \end{equation}
  thus $\rho$ is the zero distribution.  Hence, $T^\sharp$ has trivial kernel and as a linear map is injective.
  We have established all of the properties in \Cref{thm:pullback}.
\end{proof}

\subsection{Borel's theorem and a motivating example}
\subsection{Radial symmetry for $L^2(\Omega_2)$ and $\HH_0^n(\Omega_2)$}

\section{The PSF inverse problem}
\subsection{The edge-blur operator for radially symmetric PSFs}
\subsection{Variational formulation of PSF estimation}
\subsection{Infinite dimensional Bayesian formulation for PSF estimation}

\end{chapter}
