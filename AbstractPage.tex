%
%   This begins the abstract page
%
\pagenumbering{roman}                   % Setting the page numbering to begin with lowercase
\setcounter{page}{2}                    % roman numeral 2.
%
%   The first line is your name, followed by degree and completion month/year as above.
%

Joyce, Kevin, Doctorate of Philosophy, May 2016 \hfill Mathematics
   
% Self Explanatory (but not all caps as on the title page)
Point Spread Function Estimation and Uncertainty Quantification

%
Committee Chair: Johnathan Bardsley, Ph.D.    % Self Explanatory
%
\setlength{\parindent}{2ex}
%
% Now for you abstract text:
%
%\indent The first half of this dissertation focuses on computational methods for solving the constrained quadratic program (QP) within the support vector machine (SVM) classifier. One of the SVM formulations requires the solution of bound and equality constrained QPs. We begin by describing an augmented Lagrangian approach which incorporates the equality constraint into the objective function, resulting in a bound constrained QP. Furthermore, all constraints may be incorporated into the objective function to yield an unconstrained quadratic program, allowing us to apply the conjugate gradient (CG) method. Lastly, we adapt the scaled gradient projection method of \cite{Zanni} to the SVM QP and compare the performance of these methods with the state-of-the-art sequential minimal optimization algorithm and M\begin{scriptsize}ATLAB\end{scriptsize}'s built in constrained QP solver, \texttt{quadprog}. The augmented Lagrangian method outperforms other state-of-the-art methods on three image test cases.
%
%\indent The second half of this dissertation focuses on computational methods for large-scale Kalman filtering applications. The Kalman filter (KF) is a method for solving a dynamic, coupled system of equations. While these methods require only linear algebra, standard KF is often infeasible in large-scale implementations due to the storage requirements and inverse calculations of large, dense covariance matrices. We introduce the use of the CG and Lanczos methods into various forms of the Kalman filter for low-rank approximations of the covariance matrices, with low-storage requirements. We also use CG for efficient Gaussian sampling within the ensemble Kalman filter method. The CG-based KF methods perform similarly in root-mean-square error when compared to the standard KF methods, when the standard implementations are feasible, and outperform the limited-memory Broyden-Fletcher-Goldfarb-Shanno approximation method. 
%
%
\pagebreak
